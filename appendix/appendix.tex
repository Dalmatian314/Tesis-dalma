\appendix


\chapter[Cálculos adicionales del capítulo 4]{Cálculos adicionales del capítulo \ref{nanodiamonds}}
\label{apendice_ND}

%\vspace{-0.5cm}

\section{Sistema nanodiamante+PDI covalente}

\begin{figure}[!htb]
\centering
\subfloat[]{%
  \includegraphics[width=0.4\textwidth]{appendix/figs/ND_H+PDI_covalent.pdf}
  \label{ND_H+PDI_covalent}
}
\subfloat[]{%
  \includegraphics[width=0.4\textwidth]{appendix/figs/TC_covalent_spanish.pdf}
 \label{UFCT_covalent}
}
\caption{(a) Sistema donor-aceptor ND+PDI covalente junto con esquema de la perturbación láser y la transferencia de carga.(b) $\Delta q$ en función del tiempo para el donor (curva negra) y el aceptor (curva azul). La perturbación tipo pulso seno cuadrado en sintonía con la transición HOMO-LUMO del aceptor se encuentra representada en rojo.}
\end{figure}

\newpage
\section[Dependencia con la perturbación]{Dependencia de la TC con la intensidad del campo y el ancho del pulso}

\begin{figure}[!htb]
\centering
\subfloat[]{%
  \includegraphics[width=0.4\textwidth]{appendix/figs/intensity_spanish.pdf}
  \label{intensity}
}
\subfloat[]{%
  \includegraphics[width=0.4\textwidth]{appendix/figs/pulses_spanish.pdf}
 \label{pulses}
}
\caption{(a) $\Delta q$ del donor en función del tiempo para distintas intensidades de campo aplicadas (desde 0,01 a 0,10 eV\AA\textsuperscript{-1}. En todos los casos se mantiene el ancho del pulso fijo en 50 fs.(b) $\Delta q$ en función del tiempo para el donor para distintos anchos de pulso. En todos los casos se mantiene constante la intensidad del campo aplicado en 0,10 eV\AA\textsuperscript{-1}}
\end{figure}

\newpage
\section{Dependencia de la TC con la distancia y el tamaño del ND}

\begin{figure}[!htb]
\centering
\subfloat[]{%
  \includegraphics[width=0.4\textwidth]{appendix/figs/charge_vs_distances.pdf}
  \label{distances}
}
\subfloat[]{%
  \includegraphics[width=0.4\textwidth]{appendix/figs/sizes_spanish.pdf}
 \label{sizes}
}
\caption{(a)Carga final del donor en función de la distancia entre el donor (nanodiamante) y el aceptor (PDI). El eje de las ordenadas tiene escala logarítmica. En todos los casos, el ancho del pulso fue de 50 fs y la intensidad de campo 0,10 eV\AA\textsuperscript{-1}.(b) $\Delta q$ en función del tiempo para el donor para nanodiamantes de distintos tamaños acoplados con PDI. En todos los casos se mantiene constante el ancho del pulso en 50 fs y la intensidad del campo aplicado en 0,10 eV\AA\textsuperscript{-1}}
\end{figure}

\newpage
\section[Simulaciones de Ehrenfest]{Simulaciones de Ehrenfest y dependencia con la intensidad del campo}


\begin{figure}[!htb]
\centering
\includegraphics[width=0.5\linewidth]{appendix/figs/TC-ehren_fields_spanish.pdf}
\caption{$\Delta q$ del donor en función del tiempo para distintos valores de intensidad de campo (en eV\AA\textsuperscript{-1}) para el sistema ND+PDI cuando los núcleos son móviles.}
\label{ehren_fields_spanish}
\end{figure}
