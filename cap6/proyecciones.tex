\section{Proyecciones}
\label{proyecciones}

%Como se desarrolló en el capítulo \ref{vdW_Photovoltaics}, 
A partir de los experimentos publicados por Carlos A. Palma en el año 2016 sobre sensibilización de grafeno a partir de redes supramoleculares
autoensambladas, se generó una nueva rama de investigación en este tipo de materiales fotoactivos con potencial aplicación en el campo de las celdas
solares orgánicas. La principal novedad de los experimentos reside en el control molecular de la nanomorfología de la capa activa (algo incontrolable en las OSC tipo BHJ que se utilizan actualmente). Además, el sustrato (grafeno) donde se depositan las monocapas sensibilizantes tiene propiedades conductoras, ahorrando pasos en la recolección de los portadores de carga, y es transparente, lo que permite pensar en arquitecturas tipo tandem apilando capas de distintos materiales para aumentar la eficiencia de las celdas. En este contexto, las simulaciones TD-DFTB de sistemas periódicos y la experiencia del grupo en simulaciones fotodinámicas presentan una oportunidad invaluable en la comprensión del funcionamiento de este nuevo tipo de arquitectura de OSC. Por un lado, el reciente desarrollo e implementación de sistemas periódicos en la dinámica electrónica en el código DFTB+ permite el estudio de este tipo de sistemas desde un enfoque híbrido, donde algunas direcciones son tratadas como periódicas y otras no. Este enfoque tiene la desventaja de requerir superceldas muy grandes para poder describir correctamente las propiedades del sistema, aumentando el costo computacional con N\textsuperscript{3}. Por ello, para disminuir este costo y poder simular estos sistemas, será necesaria la implementación completa de los puntos k en el código, sumada al cambio de {\it ``gauge''} de distancia a velocidad. Este último requisito es necesario para el adecuado tratamiento de la electrodinámica en sistemas infinitos. De esta manera, se podrán simular estos materiales con una descripción acertada de sus propiedades optoelectrónicas utilizando celdas unidad con condiciones periódicas de contorno en lugar de superceldas, bajando considerablemente los tiempos de cálculo considerablemente. Estos desarrollos y simulaciones están planteados a realizarse en un futuro en conjunto con el grupo de Bremen desarrollador del código DFTB+.
Por otro lado, la arquitectura que plantean este nuevo tipo de celdas implican una ventaja para la simulación desde el punto de vista práctico. El hecho de que las moleculas sensibilizadoras estén en contacto directo con la plataforma conductora, implica que no se requiere de la difusión excitónica y subsiguientes etapas descripta en el capítulo \ref{BHJ} para la colección de los portadores de carga. Las moléculas están inyectando carga directamente en el ánodo. A nivel computacional, esto significa que se podrían calcular parámetros de generación de corriente fotoinducida comparables directamente con el experimento, ya que la generación de corriente solo dependería de una etapa.


