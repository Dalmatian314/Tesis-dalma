\section{Conclusiones}

En el desarrollo del presente trabajo de tesis se han presentado resultados para la transferencia de carga fotoinducida en una serie de sistemas. En cada capítulo se describieron conclusiones específicas al sistema correspondiente. En el presente capítulo se presenta un modelo simplificado que puede servir como un unificador transversal al poder describir las características de cada uno de los sistemas descriptos anteriormente de una manera más intuitiva. Si bien los sistemas estudiados presentan estructuras atomísticas muy diferentes, su estructura electrónica es similar. En todos los casos están formados por un colorante que se fotoexcita y actúa como donor o aceptor de carga hacia un material nanoestructurado como son las nanofibrillas presentadas en el capítulo \ref{BHJ}, el nanodiamante presentado en el capítulo \ref{nanodiamonds} o la nanocinta de carbono en el capítulo \ref{vdW_Photovoltaics}. El colorante que es fotoexcitado presenta estados localizados y energías de HOMO y LUMO bien definidas. Los materiales nanoestructurados (que llamaremos {\em sustrato} en adelante) presentan una alta densidad de estados que puede estar dividida por un \emph{gap} (como es el caso de las nanofibrillas o el nanodiamante) o ser metálico (o con un \emph{gap} muy pequeño) como es el caso de las nanocintas de carbono.

\begin{figure}[!htb]
\centering
\subfloat[]{%
  \includegraphics[width=0.3\textwidth]{cap6/figs/figs_cgs_1.pdf}
  \label{modelitoa}
}
\subfloat[]{%
  \includegraphics[width=0.3\textwidth]{cap6/figs/figs_cgs_2.pdf}
 \label{modelitob}
}
\subfloat[]{%
  \includegraphics[width=0.3\textwidth]{cap6/figs/figs_cgs_3.pdf}
 \label{modelitoc}
}
\caption{Modelos posibles para un colorante acoplado a un cuasicontinuo o continuo. En el caso (a) el cuasicontinuo está desocupado, en el caso (b) ocupado. El caso (c) representa un continuo metálico que se encuentra ocupado hasta un nivel intermedio entre los estados fundamental y excitado molecular.}
\label{modelitos}
\end{figure}

En la figura \ref{modelitos} se describe en forma esquemática un modelo simple para la estructura electrónica de los sistemas estudiados. Los estados HOMO y LUMO de la molécula tienen energías bien definidas y están representados por líneas denotadas como $|g\rangle$ y $|e\rangle$ respectivamente. La posibilidad de fotoexcitación resonante a una energía determinada implica la existencia de un acoplamiento dipolar entre los estados dado por el momento dipolar de transición $\mu_{ge}$. Se describen tres situaciones posibles. En la figura \ref{modelitoa} el LUMO está acoplado a un cuasicontinuo de estados del {\em sustrato}, en la figura \ref{modelitob} es el HOMO el que se encuentra acoplado. Estos esquemas corresponden a las situaciones en las que el colorante está acoplado a una nanofibrilla o nanodiamante. En la figura \ref{modelitoc} ambos orbitales HOMO y LUMO están acoplados al continuo. La magnitud del acomplamiento está dada por el elemento de matriz $V_{g,l}$ o $V_{e,l}$ según corresponda, entre el estado fundamental o excitado y un estado electrónico $|l\rangle$ perteneciente al cuasicontinuo o continuo. Un modelo similar al que aquí se describe es utilizado para modelar el ensanchamiento espectral de moléculas en solventes \cite{Nitzan2006}.

Estos diagramas representan situaciones muy generales que no sólo describen los sistemas estudiados en esta tesis sino también sistemas colorante-nanopartícula en donde el sustrato es un óxido metálico \cite{Cande2018,Marquez2018,Negre2012,Oviedo2012}, un complejo de una doble hebra de DNA que se encuentra enlazada por un hilo de plata \cite{Berdakin2016,Berdakin2017} y otros que han sido estudiados en el grupo de trabajo en el que se desarrolló esta tesis. En todos estos sistemas se observa que la fotoexcitación del colorante produce una transferencia de carga que se manifiesta como un cambio sostenido de las cargas del sustrato que crece en forma aproximadamente lineal en el tiempo. La escala de tiempo en la que se da la transferencia de carga es desde algunas decenas a algunos cientos de femtosegundos. Si bien en todos los casos la transferencia de carga puede ser catalogada como ``ultrarrápida'', en los hechos se observa que estas escalas difieren en aproximadamente dos órdenes de magnitud. Esta escala de tiempo es, sin embargo, mucho mayor a la escala de tiempo de oscilación del campo eléctrico que fotoexcita al sistema, y que corresponde a la diferencia de energía entre HOMO y LUMO. Esta afirmación no es del todo correcta ya que en TD-DFTB, existe una renormalización dinámica de la energía de excitación por la interacción electrón-electrón que hace que la energía de la excitación no corresponda a la diferencia de energía entre el HOMO y el LUMO, sin embargo, podemos pensar que existen dos autoestados del sistema que corresponden al fundamental y excitado de la molécula y que son los que están representados en los diagramas de la figura \ref{modelitos}.

En la figura \ref{modelitos} se representa la estructura electrónica del sistema no perturbado, es decir, sin tener en cuenta el efecto de la interacción colorante-sustrato. Aún en esta figura no perturbada es posible ya sacar algunas conclusiones. En la figura \ref{modelitoa} el estado fundamental molecular está ocupado y tanto el estado excitado como el cuasicontinuo están desocupados. Una excitación promueve electrones desde el estado $|g\rangle$ al estado $|e\rangle$. El cuasicontinuo desocupado puede actuar como aceptor de los electrones que se promueven al estado excitado. De esta manera se transferirán electrones al {\em sustrato}. Este es el caso de las DSSC estudiadas en las referencias \cite{Cande2018,Marquez2018,Negre2012,Oviedo2012} donde se produce la fotoinyección de electrones a nanopartículas de óxidos metálicos. En la figura \ref{modelitob} el estado fundamental molecular está ocupado y el estado excitado desocupado pero el cuasicontinuo está ocupado. Cuando se produce una excitación de electrones desde el estado $|g\rangle$ al estado $|e\rangle$ el cuasicontinuo ocupado puede actuar como donor de electrones que se transfieren al estado fundamental $|g\rangle$. De esta manera se transferirán huecos al {\em sustrato}. Este es el caso de los sistemas estudiados en esta tesis, donde la fotoexcitación de los cromóforos aceptores, produce la fotoinyección de huecos en el sustrato (ya sean las nanofibrillas, los nanodiamantes o las nanocintas de carbono). Esta es una primera conclusión general que el modelo descripto en la figura \ref{modelitos} provee y nos permite una interpretación transversal a todos los sistemas estudiados. En el caso descripto por la figura \ref{modelitoc} la transferencia de huecos o electrones estará determinada por la importancia relativa del acoplamiento de los estados fundamental y excitado. Por ejemplo, el caso del sistema nanocinta de carbono sensibilizada con 6CGNR del capítulo \ref{vdW_Photovoltaics}. En este sistema, tanto el estado fundamental $|g\rangle$ como el estado excitado $|e\rangle$ se encuentran acoplados con el sustrato. Sin embargo, se observa fotoinyección de electrones al sustrato debido seguramente a una mayor interacción del estado $|e\rangle$ con el mismo. Lo que coincide con la naturaleza química del cromóforo donor.

\begin{figure}[!htb]
\centering
\includegraphics[width=0.6\textwidth]{cap6/figs/figs_cgs_4.pdf}
\caption{Estructura electrónica del sistema mostrado en la figura \ref{modelitoa} una vez tenida en cuenta la perturbación representada por los lementos de matriz $V_{el}$}
\label{modelitod}
\end{figure}

Si se diagonaliza el Hamiltoniano del sistema incluyendo la perturbación de los elementos de matriz $V_{el}$ en el modelo presentado en la figura \ref{modelitoa}, el estado $|e\rangle$ se mezcla con los estados $|l\rangle$ y adquiere un ancho en energía. Una situación similar puede plantearse para los otros dos modelos presentados en la figura \ref{modelitos}. La transición fotoinducida por el campo, cuya probabilidad está determinada por el elemento de matriz de momento dipolar $\mu_{ge}$, ahora se distribuye sobre el conjunto de estados $|e,l\rangle$. Esto ocasiona una disminución de la intensidad del pico de absorción molecular y un enzanchamiento. Estos efectos serán más o menos importantes de acuerdo a la magnitud del acoplamiento entre el estado molecular y el cuasicontinuo o continuo. En todos los sistemas estudiados en esta tesis el acoplamiento entre el colorante y el {\em sustrato} se debe a interacciones débiles, de van der Waals. La interacción electrónica es por lo tanto muy baja y, si bien se observan disminuciones de intensidad y corrimientos en todos los casos, se conserva la identidad del estado molecular como se desprende de los espectros de absorción obtenidos.

La interacción con el sustrato genera una pérdida de identidad del estado molecular que se acopla y se transforma a su vez al cuasicontinuo o continuo. Esta pérdida de identidad es la responsable de la existencia del proceso de transferencia de carga. La fotoexcitación ahora no se puede ver como una única excitación sino como un cuasicontinuo o continuo de excitaciones desde el estado fundamental a un conjunto de estados excitados. En cada uno de estos estados excitados, los estados $|l\rangle$ del sustrato tienen una participación y por lo tanto la excitación directa de electrones desde el estado fundamental a cada uno de estos estados provoca una probabilidad no nula de encontrar al electrón excitado en un estado del sustrato. Es ésta la causa de la transferencia de carga, una transferencia directa de probabilidad de encontrar electrones desde el estado fundamental a estados del sustrato a través de su acoplamiento con el colorante. La velocidad de transferencia de carga estará determinada por la probabilidad de excitación (relacionada con la intensidad luminosa y el momento dipolar de transición de la excitación molecular) pero también por el acoplamiento colorante sustrato.

Como se enunció más arriba, la escala de tiempo de la transferencia de carga es mucho menor a la del proceso de fotoexcitación. Esto se debe a que la interacción entre el sustrato y el colorante es débil. Esta característica de la interacción es fundamental para asegurar un proceso de transferencia de carga efectivo. Si la interacción entre ambos componentes fuera fuerte, la transferencia de carga aumentaría su velocidad, pero también lo haría el proceso inverso. El hecho de que la velocidad de transferencia sea lenta, acoplada a la alta densidad de estados provista por el sustrato permite que este actúe como un sumidero para la carga fotoexcitada. Claramente las situaciones descriptas en las figuras \ref{modelitob} y \ref{modelitoc} pueden razonarse de forma análoga.

En el caso de carga transferencia al nanodiamante, tal y como se describió en el capítulo correspondiente, la situación agrega una nueva dimensión. Al ser el sustrato un material de extrema dureza química, su respuesta al cargado es muy fuerte. Lo que causa una desintonización que da lugar al efecto ``puerta trampa'' allí descripto y que provee una fuente adicional de irreversibilidad e incremento de la eficiencia.

En síntesis, es posible encontrar una figura esquemática de la estructura electrónica que permite explicar los resultados expuestos a lo largo del presente trabajo de tesis. La transferencia de carga fotoinducida, al menos en los sistemas estudiados, requiere de un colorante acoplado débilmente a un continuo o cuasicontinuo. Las energías relativas entre el HOMO y LUMO en relación al sustrato determinan la dirección de la transferencia de carga. La interacción débil es necesaria para que el sustrato actúe de forma efectiva como un sumidero.

 
%\section{Proyecciones}
\label{proyecciones}

%Como se desarrolló en el capítulo \ref{vdW_Photovoltaics}, 
A partir de los experimentos publicados por Carlos A. Palma en el año 2016 sobre sensibilización de grafeno a partir de redes supramoleculares
autoensambladas, se generó una nueva rama de investigación en este tipo de materiales fotoactivos con potencial aplicación en el campo de las celdas
solares orgánicas. La principal novedad de los experimentos reside en el control molecular de la nanomorfología de la capa activa (algo incontrolable en las OSC tipo BHJ que se utilizan actualmente). Además, el sustrato (grafeno) donde se depositan las monocapas sensibilizantes tiene propiedades conductoras, ahorrando pasos en la recolección de los portadores de carga, y es transparente, lo que permite pensar en arquitecturas tipo tandem apilando capas de distintos materiales para aumentar la eficiencia de las celdas. En este contexto, las simulaciones TD-DFTB de sistemas periódicos y la experiencia del grupo en simulaciones fotodinámicas presentan una oportunidad invaluable en la comprensión del funcionamiento de este nuevo tipo de arquitectura de OSC. Por un lado, el reciente desarrollo e implementación de sistemas periódicos en la dinámica electrónica en el código DFTB+ permite el estudio de este tipo de sistemas desde un enfoque híbrido, donde algunas direcciones son tratadas como periódicas y otras no. Este enfoque tiene la desventaja de requerir superceldas muy grandes para poder describir correctamente las propiedades del sistema, aumentando el costo computacional con N\textsuperscript{3}. Por ello, para disminuir este costo y poder simular estos sistemas, será necesaria la implementación completa de los puntos k en el código, sumada al cambio de {\it ``gauge''} de distancia a velocidad. Este último requisito es necesario para el adecuado tratamiento de la electrodinámica en sistemas infinitos. De esta manera, se podrán simular estos materiales con una descripción acertada de sus propiedades optoelectrónicas utilizando celdas unidad con condiciones periódicas de contorno en lugar de superceldas, bajando considerablemente los tiempos de cálculo considerablemente. Estos desarrollos y simulaciones están planteados a realizarse en un futuro en conjunto con el grupo de Bremen desarrollador del código DFTB+.
Por otro lado, la arquitectura que plantean este nuevo tipo de celdas implican una ventaja para la simulación desde el punto de vista práctico. El hecho de que las moleculas sensibilizadoras estén en contacto directo con la plataforma conductora, implica que no se requiere de la difusión excitónica y subsiguientes etapas descripta en el capítulo \ref{BHJ} para la colección de los portadores de carga. Las moléculas están inyectando carga directamente en el ánodo. A nivel computacional, esto significa que se podrían calcular parámetros de generación de corriente fotoinducida comparables directamente con el experimento, ya que la generación de corriente solo dependería de una etapa.














