\chapter*{Agradecimientos}
\addcontentsline{toc}{chapter}{Agradecimientos}

Mientras escribo estas lineas, en un par de horas comienza el toque de queda en Chile. Un toque de queda en democracia. Mientras escribo esto, en Chile el pueblo está en las calles reclamando por una sociedad más justa e igualitaria. Más de 30 años de continuas políticas neoliberales colocan a Chile entre las sociedades más desiguales de Latinoamérica. Donde la mercantilización de los derechos sociales ha llevado por ejemplo a tener una de las educaciones superiores más caras del mundo, mientras acá la educación pública es gratuita. Una protesta social iniciada por estudiantes de secundario que no les tienen miedo a los militares, a quienes se fueron sumando profesores, laburantes y todo el pueblo chileno. Y el presidente Sebastián Piñera salió a decir ``Estamos en Guerra'' soltando a los militares a la calle. El número de muertos ya asciende a 18. Parece que este tipo está en guerra con su propio pueblo. ``Chile es el modelo a seguir'' decían algunos de este lado de la coordillera... Me permito escribir estas líneas para que valoremos la educación pública y gratuita que tenemos en Agentina, que hemos ganado gracias a nuestras luchas históricas por ese derecho y que en estos últimos años se ha puesto en tela de juicio por algunos dirigentes políticos que ni vale la pena nombrar. También para que no hagamos oídos sordos a lo que está pasando con nuestros hermanos chilenos y abracemos y apoyemos su reclamo por una sociedad más justa. Basta de neoliberalismo en Latinoamérica. 
Un fuerte abrazo y mi apoyo a mis amigos y a todo el pueblo Chileno \#ChileDespertó \#RenunciáPiñera.

Volviendo al motivo original de estas lineas, quiero agradecer a mi director, que más que director yo lo considero un Gurú. Aquel maestro espiritual que además de mostrarte el camino y enseñarte su sabiduría, es capaz de escucharte y darte los mejores consejos tanto en el plano académico como en el personal. Una persona con una capacidad intelectual increíble pero a su vez con una humildad diría que hasta rara en su especie. Gracias por creer en mi y por toda la ayuda propiciada estos años. Es realmente un gusto trabajar con vos. Además, quiero agradecerle a Lourdes por tratarme como a un hijo en varias ocasiones. En particular cocinandomé en su propia casa mientras yo terminaba de escribir la primera versión de la tesis. Gracias infinitas. Claramente esto no hubiera sido posible sin ustedes. Les quiero mucho.

Gracias a los miembros de la comisión asesora, Gustavo, Marcelo y Ezequiel por las discusiones durante el desarrollo de la tesis y el interés genuino en el trabajo. Gracias al Dr. Busnengo por aceptar la terea de evaluar esta tesis. A Carlos Palma, por su entusiasmo científico y su constante motivación para trabajar juntos. A Thomas Fraeunheim, por invitarme a una estadía en Bremen y ofrecerme la posibilidad de encarar un proyecto en su grupo de investigación.

Quiero agradecer a la escuela primaria Gobernador Álvarez, por aquellos primeros pasos en mi formación inicial y en particual a la maestra Teresita Salli por contarme de la existencia de la Esc. Sup. de Com. Manuel Belgrano en los últimos meses de quinto grado. Por envalentonarme a mí y a mi vieja para rendir el examen de ingreso. Por supuesto que yo no entendía absolutamente nada en ese momento de lo que significaba esa decisión tan importante en mi vida. 

A la Esc. Sup. de Com. Manuel Belgrano por enseñarme a pensar de una manera crítica y a no conformarme con las realidades que me parecen injustas. En particular agradecer a mi profe de Biofísica Silvana Durilem por mostrarme su pasión por la ciencia y sus clases y explicaciones super originales con todo el amor de quien adora la profesión de docente. A mi profe de Derecho ambiental y Técnicas de Laboratorio Norma Pollet, quien por un lado me introdujo al fascinante e infinito mundo de la química orgánica y por el otro me enseñó la importancia del medio ambiente y nuestro compromiso como ciudadanos en defenderlo. A mi profe de epistemología Aldo Chávez, por presentarnos a personajes como Kuhn, Lakatos y Feyerabend y hacernos pensar en el ¿Por qué? ¿Para qué? ¿Por quién? y ¿Para quién? de la ciencia. Preguntas que me parecen extremadamente necesarias en nuestra labor como profesionales y al día de hoy, no consigo entender cómo es posible que no formen parte de la currícula de las carreras en nuestra facultad. Por último a mi profe de historia Alejandra Amuchastegui, por contagiarme su amor por la historia, enseñarnos a analizar los acontecimientos en sus contextos históricos, a no caer en el facilismo de ``buenos y malos'' y a no olvidar nuestra historia para luchar siempre por un futuro más justo.

A la Universidad Nacional de Córdoba, que ya no se si es mi segunda casa o la primera. Gracias por la educación pública y gratuita de excelencia que me brindaron desde que tengo 11 años, en mi paso por el secundario, la licenciatura en Química y ahora el Doctorado. Gracias también a la Facultad de Ciencias Químicas que además de brindarme la formación de grado y posgrado solventó económicamente proyectos de extensión y articulación universitaria en los que me ha tocado participar como miembro del equipo y también como co-director. Es sumamente importante que la Facultad apoye y reconozca este tipo de actividades que promueven la co-construcción del conocimiento con otros actores sociales fuera del ámbito universitario.

A la incubadora de empresas de la UNC, por avalar, apoyar y darnos el espacio para nuestro emprendimiento Quantum Dynamics que junto con Cristián, Franco y Cande, iniciamos al comienzo de mi doctorado. Una experiencia fantástica y llena de aprendizajes. Celebro que tengamos una incubadora para generar empresas de base tecnológica a partir de ideas originales que salen de nuestra propia Universidad. 

Agradezco a las políticas públicas implementadas por el estado nacional presente hasta el 2015, apostando a la Universidad pública y gratuita de calidad y al sistema científico tecnológico. En particular agradezco la beca bicentenario que me permitió estudiar sin tener que trabajar durante la licenciatura, 2 becas de estímulo a la vocación científica durante los últimos años de la carrera, y la beca doctoral de CONICET.


A mi psicólogo Victor Hugo Campos, quien me inició en el extraordinario mundo de la terapia cognitiva y me brindó las herramientas para ``convivir con lo que soy para ser quien quiero ser''. Gracias por inciarme también en la filosofía Zen, la meditación, y por los innumerables intercambios de libros, películas, política y ciencia. Es realmente un gusto ir a terapia con vos.

A mi profe de percu africana el Sopa, por compartir su talento y conocimiento con la humildad y alegría que lo caracteriza. A mi profe de salsa y bachata Migue Flores, por enseñarme, apostar en mí y entrenarme en el hermoso mundo de la danza. A la mejor compañera de baile que uno podría tener, Romi Silva. Gracias por todo lo compartido y por las coreos juntos :) y perdón por abandonarte este último año! Les quiero mucho. A mi profe de alemán Naty Yáñez, la mejor profe del Goethe Institut sin duda. Gracias por la paciencia, los juegos, las clases y los consejos para aprender este idioma tan hermoso que es el alemán y que me abrió tantas puertas para mi desarrollo profesional y me permitió viajar 2 veces a Alemania. Sin duda no lo hubiese logrado sin vos.

A mis amigos del secundario, la banda del pretzel (corriente fundadora): el enano (Nahuel), Larry (Joa) y la tota (Ángel), con quienes sobrevivimos al hostigamiento constante que sufrimos durante los primeros 3-4 años del secundario (Si... éramos unos frikis... ojo... no estoy justificando el bullying), hasta que pegamos el estirón y equiparamos alturas con nuestros compañeros más grandes (bueno, todos menos el enano que nunca creció, pero a esa altura ya lo podíamos defender nosotros). También a la banda del pretzel-XL, donde se sumaron Cani (Otto), el Vikingo (Gabi), Bombi (Flor) y la Flany (Dani). Gracias por todos los hermosos momentos vividos juntes.

A mis amigos de la facu, los somostangilesjuntxs: el Lobito, la Vane, la Dani, el Lémur, el Mati Liendro, la Romi, Biancófilo, la Guada, el Crico y Crushka. Ah! Y cómo olvidarme de Nery Furtado. Claramente la carrera no hubiera sido la misma sin ustedes. Hermoso grupo. Les quiero a montones. A la Dani, por ser una amiga super tierna (rwau) que siempre está cuando se necesita. Es esa oreja perfecta capaz de escucharte en los momentos que más hace falta y te da los mejores consejos. Además de compartir cierto grado de locura en la que nos reímos de todo (tcha... Dani...). Al lobo, por ser prácticamente un hermano desde que nos conocimos en la facu. Un chabón con el corazón tan grande que no le cabe en el pecho. Otro con el que nos hartamos de pelotudear sanamente y compartir momentos tanto difíciles como gratos y llenos de felicidad. Sé que últimamente no nos vemos tanto pero también sé que siempre cuento con vos hermano, de eso no tengo duda. Acá también quiero agradecer a tu vieja, la Ale, quien se ha portado conmigo como una segunda madre. Te quiero mucho Ale! 

A Nirvana y Eloisa del grupo Settlers con quienes hemos compartido juegos de mesa durante nuestras etapas doctorales y forjamos amistades que persisten a la distancia al día de hoy.


A lxs olímpicxs, un hermoso grupo de científicxs que arrancamos a juntarnos con la excusa de hacer deportes y terminamos haciendo juntadas para divertirnos, viajando a los juegos deportivos de CONICET, ganando copas y hasta defendiendo nuestro derechos en las marchas (algo que parece que cada vez es más necesario). Siempre les llevo en mi corazón. A Roxy, por el amor y la compañía en nuestros años juntes. Gracias por bancarme y por todos los momentos compartidos. Sos una persona excepcional.

Quiero agradecer a mis colegas del DQTC, con los que hemos compartido mucha ciencia, asados y luchas. A Belén, por creer en mí, por la paciencia y por ayudarme tanto en los primeros pasos de este camino. Al tío socialista Luis, por ser una de las personas más cultas que conozco y compartir su sabiduría constantemente, gracias también por el apoyo incondicional y el confiar en nosotros para actualizar la guía de mate 2 e intentar hacerla más pedagógica, más práctica y divertida para les alumnes. A Germán, por su amistad y por siempre estar dispuesto a ayudarme y compartir su conocimiento. Y también por aquella campera que me salvó la vida en mi primer viaje a Alemania. Al Rodri Quiroga por las interminables charlas políticas y su compromiso con el gremio de ADIUC, siempre defendiendo los derechos de les docentes e investigadores universitarios. Al box de becaries, que en este último tiempo han estado presente compartiendo lindos momentos, charlas y mates siesteros, Juan (the real one), el José, Tintín, la Dalmatian, Nathamiel y la Elu.

Al grupo de dinámica cuántica, Cande, Franco, Dalmatian, Mati, Oscar, Tincho T. y Fede. Al Fede por el intercambio científico, el bardo continuo, las risas y los momentos de lucha compartidos. A la Dalma por todos los momentos compartidos, las bardeadas, los proyectos juntos, por tu amistad, por bancarme hasta cuando no me banco ni yo y por los peores mates que tomé en mi vida (aaaAAGGHH). Na mentira, hay peores. A la Cande porque además de ser una excelente compañera de trabajo es una amiga excepcional, de esas personas con las que podés contar en cualquier momento. A Franco, que ya más que un amigo es un hermano para mi. Una persona increíble, con un talento y una pasión por la ciencia que contagia e inspira. Pero a su vez con una visión social y empuje por cambiar las cosas que motivan a cualquiera que esté a su lado. Una persona además con la que literalmente me cago de risa y compartimos momentos de ocio del tipo excelentes. No hace falta mencionar las toneladas de carcajadas DIARIAS que nos produce la bastardización cannabis mediadas ahora por la red del tipo Schwifi (lamao). Además, es una persona que siempre estuvo en los momentos difíciles y me acompañó incondicionalmente. 

A Neko y Kiefer, dos grandes personas que la vida me cruzó en Berlín y que en muy poco tiempo se ganaron un espacio en mi corazón. A Carol, por los hermosos e intensos momentos vividos juntos.

A los locos de la casa de barrio JardínS, el vikingo, LujánS y el negro Mocte (¿Qué te pensás chenegra?). Si esa casa hablara muchaches... AYyy.. BAAaaAAssta. Por los hermosos momentos vividos en la casa y la hermandad que generamos en tan poco tiempo. Por aquellas cenas compartidas, las tardes de música y los bardos constantes. Siempre es lindo llegar a casa y que haya ruido. Les quiero y extraño mucho. A Pablo, mi amigo-hermanito mayor, mi profe de Basquet en el cole, mi profe de guitarra, una persona que quiero mucho. A la Yadi y la Pao, dos personas increíbles con la que nos tocó convivir en córdoba durante un mes y medio y que quiero un montonazo. Sé que pronto nos volveremos a ver.

A la juguetería Mis Chiches, fuente de trabajo en aquellas semanas de vacaciones durante la carrera. En particular a Nora y Matu, quienes han sido como madres postizas para mi. Gracias por todo el cariño, los helados y los mates compartidos en el negocio.

Por último quiero agradecer a mi familia. A mis tías Victoria y Negra por brindarme tanto amor y apoyo incondicional desde pequeño. A mi tío Toti, por ser mi abogado de referencia, por enseñarme montones de cosas y por estar siempre. A mi abuelo, que aunque ya no esté con nosotros lo llevo siempre en mi corazón. Él hizo de padre durante los primeros años de mi vida. A mi tío Eduardo que también estuvo muy presente en mi infancia haciendo las veces de tío y las veces de hermano mayor cuando peleábamos. 
A mi viejo, que llegó cuando yo tenía 6 años y se quedó para siempre. Gracias por bancarme en todas las que me propuse y apoyarme siempre. Y gracias por bancar y ayudar a mi vieja, porque criar sola no es para nada fácil. Gracias a mi vieja que me dió todo. Estoy seguro de que más de una vez, no comió ella para darme a mi. Fuente de amor puro e incondicional, jamás me dijo lo que tenía que hacer y siempre me dió la libertad de elegir lo que yo quiera para mi futuro. Me acompañó y me acompaña siempre en cada una de las decisiones que tomo. Y me ayuda siempre como pueda. Siempre está pendiente de mi y de mi bienestar. Por ese cuidado y amor incondicional en la vida, muchísimas gracias vieja. Te amo.

\vspace{0.5cm}
A todes les que hicieron esta tesis posible, GRACIAS.


